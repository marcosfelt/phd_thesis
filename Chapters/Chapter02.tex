%*****************************************
\chapter{Background}\label{ch:background}
%*****************************************

In this chapter, I first review transfer learning and then the two type of machine learning used int his thesis: Bayesian optimization and molecular property prediction.

\section{Transfer Learning}

% Inductive bias in machine learning
The key idea behind transfer learning is to improve the inductive bias of a machine learning model. Inductive bias is the inherent assumptions the model makes about the data it is being trained on. Therefore, inductive bias can help a model generalize from a small dataset to a large set of examples. Transfer learning creates a better inductive for a model using data from a relevant dataset. In this section, we will cover two methods for transfer learning that are explored in this thesis. 

\subsection{Multi-task Learning}

Multi-task learning is a form of transfer learning that builds one model to predict several related tasks. By leveraging the similarity between the tasks, the multi-task model often achieves more accurate predictions than individual models trained on each task.\cite{Simes2018} As a form of transfer learning, multi-task learning is particularly useful when the amount of data available for each task is limited.

Previously, multi-task learning has found use in several life science applications. Ramsundar and co-workers Ire the first to complete a detailed study of using multi-task deep learning for predicting the activity of molecules in various biological assays.\cite{Ramsundar2015} They found that increasing the number of tasks results in better performance in most cases, primarily due to active compounds sharing similar mechanisms across tasks. The same authors extended this work by comparing several different deep learning architectures techniques to the gold standard random forests on different pharmaceuticals datasets. They again found that multi-task learning is very effective at improving over random forests, though not always.\cite{Ramsundar2017} In contrast, Sadawi compared various types of multi-task learning in several QSAR datasets and found that random forests often outperformed deep learning models.\cite{Sadawi2019} These results suggest that performance of multi-task learning is highly dependent on the task.

More recent work has examined how multi-task learning can be used for chemical reactions. Struble et al. used a multi-task deep learning model to predict site selectivity of C-H activation reactions, resulting in a accuracy improvement compared single task models.\cite{Struble2020} The accuracy improvement of the multi-task model was attributed to several of the tasks having a small number of examples but high similarity to other tasks. In another study, a multi-task transformer model was used to predict the outcomes of carbohydrate reactions.\cite{Pesciullesi2020} The authors found that training in multi-task mode instead of sequential fine-tuning led to loIr error on the test set. 

These studies demonstrate that multi-task learning is a potentially promising method to improve the performance of models in the life sciences. However, before the work in this chapter, there was no application of multi-task learning to reaction optimization. Here, I use Bayesian optimization, which requires probabilistic models. While both random forests and deep learning models can be modified to give probabilistic results,\cite{Ling2017, Duan2020, Soleimany2021} their calibrated uncertainties are not excellent in low data regimes. In contrast, GPs, which are commonly used as probabilistic models in Bayesian optimization, offer calibrated uncertainties as a standard feature (see chapter one), making them the best choice for the application of reaction optimization. The challenge is in adapting GPs for multi-task prediction.

\subsection{Pretraining}

Pretraining is a method where a neural network is trained on auxiliary dataset prior to training on the main dataset. There are two types of transfer learning tasks self-supervised tasks and representation learning. Self-supervised learning uses partial data to obtain the whole dataset. Alternatively,  supervised pretraining works by training on a similar task, freezing weights of the network, and then trained.

\section{Bayesian optimization}

Bayesian optimization aims to solve the optimization problem.
\begin{equation}
    \max_x y(x)
\end{equation}
where $y(x)$ is the underlying function that are observed via experiments. Bayesian optimization (BO) achieves this optimization by first training a surrogate probabilistic model to represent the underlying function and, then, optimizing an acquisition function informed by this model to choose next experiments. Typically,  the probabilistic model is a Gaussian Process (GP). 

\subsection{Gaussian Processes}

A GP is a stochastic process characterized by a mean function $\mu(x)$ and covariance function $k_{\theta}(x,x')$.\cite{Rasmussen2006} The covariance function is often called a kernel, which is the term I will use henceforth.

\begin{equation}
    f(x)= \mathcal{GP}(\mu(x), k_{\theta}(x, x'))
\end{equation}

$\theta$ are referred to as the hyperparameters of the GP and are varied to train the GP. Given a finite set of $N$ inputs $\mathbf X = \{\mathbf x_1, \mathbf x_2, \dots, \mathbf x_N \ \vert x_i \in \mathbb R^m \}$ that correspond with outputs $\mathbf y = \{y_1, y_2, \dots y_N \vert  y_i \in \mathbb R \}$, the GP is a multivariate Gaussian distribution:

\begin{equation}
    f(\mathbf X) \sim \mathcal N(\mu_{\theta}(\mathbf X), k_{\theta}(\mathbf X, \mathbf X'))
\end{equation}

The mean function and kernel act as a prior on the GP.  $\mu_{\theta}(x)$ is usually set to zero because the kernel  $k_{\theta}(x, x')$ can expressively represent any arbitrary function. In this work, I use the Matérn 5/2 kernel, with hyperparameters $\theta=\{\sigma,\mathbf L \}$. $\sigma \in \mathbb R$ is the scaling hyperparameter and $\mathbf L \in \mathbb R^m$ is a lengthscale that indicates the significance of each input feature:
\begin{equation}
    k_{\theta}(x, x') = \sigma^2 \biggl(1 + \sqrt{5}d_{\theta}(x,x')+\frac{5}{3}d_{\theta}(x,x')^2\biggr)\exp\biggl(-\sqrt{5}d_{\theta}(x,x') \biggr)
\end{equation}
$d_{\theta}(x,x')$ is the euclidean distance weighted by the lengthscale.
\begin{equation}
    d_{\theta}(x,x')=\biggl\lVert \frac{x-x'}{L} \biggr\rVert_2
\end{equation}
Inference on the GP is done by calculating the posterior of the GP. The posterior of the GP is also a Gaussian distribution:

\begin{equation}
     \tilde f(\mathbf X) \sim \mathcal N(\tilde \mu(\mathbf X), \tilde \sigma_{\theta}(\mathbf X, \mathbf X'))
\end{equation}

\begin{equation}
    \tilde \mu_{\theta}(x) = k_{\theta}(x, \mathbf X)k_{\theta}(\mathbf X, \mathbf X')^{-1} \mathbf y
\end{equation}

\begin{equation}
    \tilde k_{\theta}(x,x') = k_{\theta}( x, x')-k_{\theta}(x, \mathbf X) k_{\theta}(\mathbf X, \mathbf X)^{-1}k_{\theta}(\mathbf X, x)
\end{equation}
where $\tilde \sigma_{\theta}(x)$ are the diagonals of the covariance matrix calculated using $\tilde k_{\theta}(x, x')$.
To train the GP, the log likelihood is maximized, which is the probability that the model predicts the training outputs given the inputs and hyperparameters. The log likelihood avoids overfitting by trading off accuracy of fit to the training data and complexity of the model.
\begin{equation}
    \log p(y \vert X, \theta) = -\underbrace{\frac{1}{2}(y-\tilde \mu_{\theta}(\mathbf X))^T k_{\theta}(\mathbf X, \mathbf X)^{-1}(y- \tilde\mu_{\theta}(\mathbf X)) }_{\text{Data  fit}}- \underbrace{\frac{1}{2} \log{\vert \tilde k_{\theta}(\mathbf X, \mathbf X) \vert} - \frac{d}{2}\log{2 \pi}}_{\text{Complexity penalty}}
\end{equation}

\subsection{Acquisition Functions}

The most commonly used acquisition function is the expected improvement (EI). In BO with EI as an acquisition function, the aim is to choose the point that is expected to improve the most upon the existing best observed point $y^* \geq \hat y(x_i) \forall i \in (1, \dots, t)$  where  $t$ is the number of observations thus far. Therefore, I create an improvement function $I(x)$ describing the improvement of the posterior of the GP over the best observed point. If there is no improvement, $I(x)=0$.
\begin{equation}
    I(x) = \max(\tilde f_{\theta}(x) -y^*, 0)
\end{equation}
For EI, I want the expectation of the improvement:
\begin{equation}
    EI(x) = \mathbb E_{y}[I(x)]
\end{equation}
After some manipulations, the following closed form for the expected improvement is found:
\begin{equation}
    EI(x) =(\tilde \mu_{\theta}(x)-\hat y^*)\Phi(Z^*) + \tilde \sigma_{\theta}(x) \phi(Z^*)
\end{equation}
where $Z^*= \frac{y^*-\tilde\mu_{\theta}(x)}{\tilde \sigma_{\theta}(x)}$.  I then solve the following optimization problem to select the next experiment $x_{next}$.
\begin{equation}
    x_{\text{next}} = \text{argmax}_{x} EI(x)
\end{equation}
Since EI has a closed analytical for the expectation and derivative, gradient based second-order optimization algorithms are typically used. 

While EI is a good baseline acquisition function, it can be overly exploitative, especially in the presence of noisy or inaccurate predictions from the GP.  Noisy EI (NEI) improves upon this by sampling the expectation of the acquisition function and the posterior probability, which reduces uncertainty in the optimum.\cite{Letham2019}. A robust and efficient impelementation of the NEI that uses a full Monte-Carlo treatment is available in the software package BOtorch \cite{Balandat2020}:

\begin{equation}
qNEI(x)= E[(\max \xi  - \max \xi_{obs} )_+]
\end{equation}

where $\xi_{obs} f(x)$ and $\xi_{obs}\sim f(x)$ are samples from the posterior of the GP.

\subsection{Transfer learning for Bayesian optimization}

One of the potential ways to accelerate Bayesian optimization is multi-task Bayesian optimization (MTBO). MTBO was originally developed to speed up hyperparameter tuning of machine learning models (e.g., choosing learning rates and batch sizes to maximize model accuracy) \cite{Swersky2013}. Using only the hyperparameters and resulting model accuracy scores of a previously trained machine learning model (which I call Task A),  MTBO decreased by up to 50\% the the number of experiments needed to find optimal hyperparameters for a new machine learning model (which I call Task B).  The data from Task A helped the model better predict Task B, even with only a few experiments for Task B.

Multi-task Bayesian optimization (MTBO) uses a multi-task probabilistic model inside a BO framework.  Swersky et al. were the first to demonstrate that multi-task BO can be effectively used to accelerate BO for hyperparameter tuning of machine learning models \cite{Swersky2013}. They demonstrated that using the results of hyperparameter tuning on one dataset could assist in tuning another with multi-task GPs. In many cases, multi-task BO could achieve 40-50\% improvements in the test accuracy of models with significantly feIr training runs.  

One of the challenges faced in applying MTBO to big data uses cases has been its lack of scalability. This in mainly due to the $O(n^3)$ cost of using GPs with exact inference. Several different approaches have been taken to solving this problem including training a vanilla neural network on the tasks and feeding the output to a Bayesian linear regression model \cite{Perrone2018} and using the auxiliary tasks to create a learned feature representation in a compressed space \cite{Hakhamaneshi2021}. These methods have retained the performance of multi-task BO while minimizing the computational cost of its deployment. However, in our case, all datasets have less than 1000 points and in many cases, less than 100. This makes multi-task GPs with exact inference more tractable.

Another direction has been to apply the machinery of multi-task GPs to multifidelity BO, which aims to leverage data from a cheap but less accurate data source to help with optimizing a more expensive function. This technique has been applied in a wide range of scenarios including parameter estimation of physics models using data from multiple experiments and simulations;\cite{Perdikaris2016} optimizing battery electrode structure using data from both cheap and expensive multitscale differential equation models (Pan 2017), and optimizing composition of alloys using a mix of DFT fidelities (Tran 2020). I foresee that, given these results, our approach will be able to be extended to combine data from simulations and experiments to rapidly optimize processes.

\section{Molecular Property Prediction}

\subsection{Molecular Parameterization}

The first step of making machine learning models that can predict molecular properties is translating moleulces to a machine-readable, typically numerical, format that can be used as input for ML models. We refer to this translation process as molecular parameterization as it aims to capture relevant molecular properties for a particular reaction. For different chemical transformations, molecular parametrizations should capture different properties such as steric hinderance of a functional group or electronegativity of neighboring atoms. The parameterization strategy should also be chosen to allow optimal compatibility with the ML model used, as prediction performance will depend on the compatibility between input format and ML model.
The baseline parameterization method for representing chemical inputs is one-hot encoding (OHE). A one (1) or a zero (0) represent the presence or absence of specified reaction components respectively - no chemical information is encoded. This approach has been shown to be effective for a variety of chemical tasks including yield prediction, but cannot extrapolate to new parts of chemical space.\cite{Pomberger2023} 

Extended-connectivity fingerprints (ECFP) is a parameterization method that captures atom types, neighboring connectivity relationships, bond types and represents the outcome in a machine-readable one-dimensional bit-vector. Circular fingerprints (e.g. Morgan fingerprints) are generated by: 1) assigning identifiers to each atom in the molecule, 2) updating each atom’s identifiers depending on the neighboring atoms, 3) removing duplicates and 4) compressing the data into a vector of set length e.g. 1024 bit (a number of zeros and ones). One of the advantages of these fingerprint based methods is that they are considered cheap features for modelling; their generation does not require vast amount of computing power/time. Yet, their ability to explicitly capture molecular properties (e.g., sterics, electronics) of molecules is limited. Typically, models that use fingerprints develop knowledge in an indirect manner, such as an implicit understanding of electronegativity associated with different halides.[254]

A much more comprehensive parameterization approach is calculating molecular descriptors using density functional theory (DFT). DFT can be used to determine the ground/excited state of molecules and thus offer fundamental insights into geometric and electronic properties.[215] As a result, DFT can be used to calculate descriptors that quantify the specific chemical properties of the given set of ligands such as the bulkiness of a molecule or electronegativity of atoms within a molecule.[255] However, DFT calculations for large libraries are often more time-consuming than actually running the corresponding reactions in a high-throughput screening format.

More recently, parameterization work has utilized neural networks to achieve the tailored nature of DFT descriptors without the computational expense. This work is divided into two approaches: natural language processing models and graph neural networks. The former leverages recent advances in language models such as transformers,\cite{Vaswani2017} where results can be achieved by training a model to predict the next word in a sentence across a wide variety of text. Since chemistry can be represented as a language in the form of simplified molecular-input line-entry system (SMILES),\cite{Weininger1988} a language model can be trained to predict the next atom in a molecule when given only a portion of the molecule, thereby saving computational expense.\cite{Schwaller2019} Since the model must understand a significant volume of chemistry to be able to predict a SMILES string, its numerical output can be used as a “learned fingerprint” for other prediction tasks.\cite{Schwaller2021}

Furthermore, the learned fingerprint can be tuned for each downstream task such as yield prediction using standard neural network training.
Alternatively, graph neural networks represent a molecule as an interconnected network of atoms and bonds. These networks can be trained to produce a “learned fingerprint” for prediction tasks. One of the most widely used forms of graph neural networks in chemistry are message passing neural networks (MPNNs), which learn relationships between neighboring atoms through iterative “messages” passed along bonds.\cite{Gilmer2017, Yang2019} MPNNs have been extended to generate fingerprints for reactions, with state of the art results. An overview of these techniques is shown in Table \ref{tab:parameterization}.

\begin{table}
    \caption{Overview of the commonly used molecular parameterization techniques for modelling chemical data.}
    \begin{tabular}{cp{0.25\linewidth}cc}
         Parameterization
    Method & Information Captured & Data Type & Example Data  \\
        \hline
         OHE & Existence/absence of a molecule & Binary encoding & [0 0 0 1 0 0 0 ] \\
         Molecular fingerprints & Atom type, atom count, chemical structure, connectivity & Binary encoding & [1 0 0 1 1 0 1 0 0 … 0 1] \\
         DFT descriptors & Inter atomic information: length, angles, volumes &  Numerical values & 0.001342, 45, $\dots$ \\
         Learned representations & Connectivity and potentially atom and bond & Numerical values & 0.001342, 45, $\dots$
    \end{tabular}
    \label{tab:parameterization}
\end{table}

\subsection{Model Architectures}

Molecules can be treated as graphs with atoms as nodes and bonds as edges. Therefore, message passing neural networks (MPNN) that operate on graphs can be used for end-to-end prediction of molecular properties \cite{Gilmer2017}.  In a MPNN, each node passes a vector (i.e., its current features) to its neighbor. This  vector is called a message. In each layer, a permutation invariant aggregation function such as the mean is used to calculate the total message for each node prior to passing through an activation function to get a new value for each node.  In contrast to traditional fixed fingerprints like ECFP \cite{Rogers2010}, the best feature vector is learned end-to-end for each property prediction task.

One type on MPNN is a directed message passing neural network (D-MPNN) in which the encoder acts on edges (bonds) instead of nodes (atoms) to improve stability of training \cite{Yang2019}. Yang, Swanson and colleagues showed D-MPNNs have superior performance to other tools such as gradient boosted trees for property prediction tasks \cite{Yang2019}. Formally, a molecule in a D-MPNN is considered to be a graph $G$ with edges $e_{vw}$ and nodes $v$ and $w$ with atom features $x_v$. A message passing update $m_v^{t}$ is as follows:

\begin{equation}
    m_v^{t+1} = \sum_{w\in N(v)} M_t(h_v^t, h_w^t, e_{vw})
\end{equation}

\begin{equation}
    h_v^{t+1} = U_t(h_v^t, m_v^{t+1})
\end{equation}

where $M_t$ is the message function, $U_t$ is the update function and $h_v^{t}$ is the hidden state at step $t$. To obtain predictions $\hat y$, the outputs of the last message passing step $T$ are passed through a feed forward network $R$ in a readout phase:
 
\begin{equation}
    \hat y = R(h_v^T \in G)
\end{equation}

In addition to using outputs of the message passing steps as input to the feed forward network, additional features $f$ can also be added:

\begin{equation}
    \hat y = R(h_v^T \in G, f)
\end{equation}

In our case of predicting activity coefficients of binary mixtures at atmospheric pressure, we treat the temperature and composition as additional features. Therefore, the feed forward network can be written as:

\begin{equation}
   \ln \gamma(x,T)= R(h_v^T \in G, x, T)
\end{equation}

Note that we predict the natural logarithm of the activity coefficient since these values can vary over an order of magnitude.