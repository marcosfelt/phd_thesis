%*****************************************
\chapter{Background}\label{ch:background}
%*****************************************

This chapter contains text published in the following paper:

\begin{enumerate}
\item Taylor C.J., Pomberger, A., \textbf{Felton, K.}, Grainger R., Barecka, M., Chamberlain, T., Bourne, R.A., Lapkin, A. A. "A Brief Introduction to Chemical Reaction optimisation". \textit{Chemical Reviews}. 2023.
\end{enumerate}

In this chapter, I first review transfer learning and then the introduce two types of machine learning used in this thesis: Bayesian optimisation and supervised learning for molecular property prediction. I discuss how transfer learning can be applied to both of these types of machine learning.

\section{Transfer Learning}

The key idea behind transfer learning is to improve the inductive bias of a machine learning model \cite{Zhuang2021}. Inductive bias is the inherent set of assumptions the model makes about the data it is being trained on. Transfer learning aims to create an inductive bias better suited to the modeling task by using auxiliary data. When effective, transfer learning can enable models to learn from smaller datasets and generalise to unseen examples more effectively. 

In transfer learning, there is a source domain and target domain. The source domain often, but not always, contains more data, while the target domain contains a smaller dataset for the primary task. Two of the most well-known examples of transfer learning are pretrained computer vision models and natural language models. Computer vision deep neural networks such as ResNet \cite{He2016} are first trained on a source task containing more than one million images (a method called pretraining) \cite{Russakovsky2015} and subsequently can learn to classify new objects on new datasets with as little as thirty images from a given target domain. Early layers in pretrained neural networks learn the fundamentals of computer vision such as identifying lines and shapes, and later layers can recognise more semantically complex concepts such as faces \cite{Zeiler2014}. These inductive biases inherent in the layers of the pretrained neural network enables the model to adapt to new tasks with smaller amounts of data.  Similarly, the revolution of large language models (e.g., the General Pretrained Transformers, GPT, models \cite{Radford2018, Brown2020}) currently happening is spurred by large-scale unsupervised pretraining of neural networks. In this approach, the source data is a large corpus of text, often extracted from the internet. The neural network is trained to predict the next (or a randomly selected \cite{Delvin2019}) word or paragraph. While conceptually simple, this task forces the model to learn the semantics of language and even some aspects of logic. As a result, when prompted with a new task in a target domain, the model can quickly adapt with a small number of examples.

The challenge of using transfer learning in process development is the lack of large source experimental datasets, so approaches that can leverage sparse and limited data are of interest. In particular, three approaches that are promising in for accelerating process development are multi-task learning, multi-fidelity learning and meta-learning.

\subsection{Multi-task Learning}

Multi-task learning is a form of transfer learning that builds one model to predict several related tasks. By leveraging the similarity between the tasks, the multi-task model often achieves more accurate predictions than individual models trained on each task \cite{Simes2018}. As a form of transfer learning, multi-task learning is particularly useful when the amount of data available for each task is limited.

Previously, multi-task learning has found use in several life science applications. Ramsundar and co-workers were the first to complete a detailed study of using multi-task deep learning for predicting the activity of molecules in various biological assays \cite{Ramsundar2015}. They found that increasing the number of tasks results in better performance in most cases, primarily due to active compounds sharing similar mechanisms across tasks. The same authors extended this work by comparing several different deep learning architectures techniques to the gold standard random forests on different pharmaceuticals datasets. They again found that multi-task deep learning performed better than random forests in many cases \cite{Ramsundar2017}. In contrast, Sadawi compared various types of multi-task learning in several QSAR datasets and found that random forests often outperformed deep learning models \cite{Sadawi2019} on significantly smaller datasets than those used by Ramsundar and co-workers; some of Ramsundar's datasets were over 100k in size while those from Sadawi and coworkers had all less than 500 examples. These results suggest that performance of multi-task learning is dependent on the relative dataset size between tasks.

Multi-task learning can also be used for chemical reactions. Struble et al. used a multi-task deep learning model to predict site selectivity of C-H activation reactions, resulting in a accuracy improvement compared to single task models \cite{Struble2020}. The accuracy improvement of the multi-task model was attributed to several of the tasks having a small number of examples but high similarity to other tasks. In another study, a multi-task transformer model was used to predict the outcomes of carbohydrate reactions \cite{Pesciullesi2020}. The authors found that training in multi-task mode instead of sequential fine-tuning led to lower error on the test set. 

Recent work has applied multi-task learning to the petrochemical industry. Schweidtmann et al. used a multi-task graph neural network to predict the ignition quality of fuels used in combustion engines \cite{Schweidtmann2020}. They found that the multi-task model gave better performance than training individual models for each property. Similar results have been found in property prediction for polymers \cite{Gurnani2023}.

\subsection{Multi-fidelity learning}

Another transfer learning technique is multi-fidelity learning, which aims to leverage data from a cheap but less accurate data source to help with predicting a more expensive task. The distinction between multi-fidelity and multi-task learning is often due to the data sources rather than the modeling techniques \cite{Folch2023, Perdikaris2016, Pan2017, Tran2020, Greenman2022}. In multi-fidelity learning, the source task often has significantly more data than the target task, while in multi-task learning, the source and target tasks often have similar amounts of data. 

In multi-fidelity learning, simulations typically serve as the source task, with experiments as the target task. For example, Greenman et al recently used multi-fidelity learning to improve predictions of optical properties of molecules \cite{Greenman2022}. They first trained a model on data from QM calculations and then used the predictions of the DFT trained model as input to a model solely trained on a limited set of experimental data. Including the predictions of the model trained on QM calculations significantly improved performance of the model. In Chapter \ref{ch:mfbo}, I explore how multi-fidelity modeling can be used in the context of controller tuning.

\subsection{Meta-learning}

Informally, meta-learning is the process of learning how to learn. The difference in this approach is the training process explicitly includes a loss function that aims to create fast learning. Most existing approaches rely primarily on learning metrics for better classification, but a recent set of methods based on optimisation are more relevant to process development \cite{Huisman2021}.  In this case, the aim is to adjust gradient based optimisation to train a model such that minimizing the loss will require the minimum number of steps. By reducing the number of steps required for optimisation, the model should be able to adapt to new cases with less data \cite{Fingerhut2017}. There are a wide variety of techniques in this family including model agnostic meta-learning (MAML) \cite{Finn2017}, Reptile \cite{Nichols2018} and Platipus \cite{Finn2018}. While I do not explore meta-learning in this thesis, it is a promising direction for future work in accelerating process development.


\section{Bayesian optimisation}

Bayesian optimisation (BO) is a method for optimising expensive to evaluate functions \cite{Shahriari2016}. In our case, these "functions" are experiments or long-running simulations. More formally, BO aims to solve the optimisation problem.
\begin{equation}
    \max_x y(x)
\end{equation}
where $y(x)$ is the underlying function that is observed via time-consuming or costly experiments. BO is designed to optimise $y$ with as few experiments as possible. It does this by first training a surrogate probabilistic model to represent the underlying function and, then, optimising an acquisition function informed by this model to choose the next experiments. Typically, the probabilistic model is a Gaussian Process (GP). 

\subsection{Gaussian Processes}

A GP is a stochastic process for modeling a function $f(x)$ characterised by a mean function $\mu(x)$ and covariance function $k_{\theta}(x,x')$ \cite{Rasmussen2006}:

\begin{equation}
    f(x)= \mathcal{GP}(\mu(x), k_{\theta}(x, x'))
\end{equation}
where $\theta$ represented the hyperparameters of the GP that are varied to train the GP. The covariance function is often called a kernel, which is the term I will use henceforth. To find the hyperparameters of the GP, we must use training data. Therefore, we can write a formulation of the GP in terms of a finite set of $N$ inputs $\mathbf X = \{\mathbf x_1, \mathbf x_2, \dots, \mathbf x_N \ \vert x_i \in \mathbb R^m \}$ that correspond with outputs $\mathbf y = \{y_1, y_2, \dots y_N \vert  y_i \in \mathbb R \}$. Given these definitions, the GP is a multivariate Gaussian distribution:

\begin{equation}
    f(\mathbf X) \sim \mathcal N(\mu_{\theta}(\mathbf X), k_{\theta}(\mathbf X, \mathbf X'))
\end{equation}

The mean function and kernel act as a prior on the GP.  $\mu_{\theta}(x)$ is usually set to zero because the kernel  $k_{\theta}(x, x')$ can expressively represent any arbitrary function. In this work, I use the Matérn 5/2 kernel \cite{Rasmussen2006}, with hyperparameters $\theta=\{\sigma,\mathbf L \}$. $\sigma \in \mathbb R$ is the scaling hyperparameter and $\mathbf L \in \mathbb R^m$ is a lengthscale that indicates the significance of each input feature:
\begin{equation}
    k_{\theta}(x, x') = \sigma^2 \biggl(1 + \sqrt{5}d_{\theta}(x,x')+\frac{5}{3}d_{\theta}(x,x')^2\biggr)\exp\biggl(-\sqrt{5}d_{\theta}(x,x') \biggr)
\end{equation}
$d_{\theta}(x,x')$ is the euclidean distance weighted by the lengthscale.
\begin{equation}
    d_{\theta}(x,x')=\biggl\lVert \frac{x-x'}{L} \biggr\rVert_2
\end{equation}
Inference on the GP is done by calculating the posterior of the GP. The posterior of the GP is also a Gaussian distribution:

\begin{equation}
     \tilde f(\mathbf X) \sim \mathcal N(\tilde \mu(\mathbf X), \tilde \sigma_{\theta}(\mathbf X, \mathbf X'))
\end{equation}

\begin{equation}
    \tilde \mu_{\theta}(x) = k_{\theta}(x, \mathbf X)k_{\theta}(\mathbf X, \mathbf X')^{-1} \mathbf y
\end{equation}

\begin{equation}
    \tilde k_{\theta}(x,x') = k_{\theta}( x, x')-k_{\theta}(x, \mathbf X) k_{\theta}(\mathbf X, \mathbf X)^{-1}k_{\theta}(\mathbf X, x)
\end{equation}
where $\tilde \sigma_{\theta}(x)$ are the diagonals of the covariance matrix calculated using $\tilde k_{\theta}(x, x')$.
To train the GP, the log likelihood is maximised, which is the probability that the model predicts the training outputs given the inputs and hyperparameters $\theta$. The hyperparameters are determined by maximising the log likelihood  shown in Equation \ref{eq:log-likelihood}.  The log likelihood avoids overfitting by trading off accuracy of fit to the training data and complexity of the model.
\begin{equation}
\begin{split}
    \log p(y \vert X, \theta) = & -\underbrace{\frac{1}{2}(y-\tilde \mu_{\theta}(\mathbf X))^T k_{\theta}(\mathbf X, \mathbf X)^{-1}(y- \tilde\mu_{\theta}(\mathbf X)) }_{\text{Data  fit}} \\
    & - \underbrace{\frac{1}{2} \log{\vert \tilde k_{\theta}(\mathbf X, \mathbf X) \vert} - \frac{d}{2}\log{2 \pi}}_{\text{Complexity penalty}}
\end{split}
\label{eq:log-likelihood}
\end{equation}

\subsection{Acquisition Functions}

The acquisition function guides the optimisation by determining the value of potential new experiments given predictions from the GP \cite{Shahriari2016}. The most commonly used acquisition function is the expected improvement (EI). In BO with EI as an acquisition function, the aim is to choose the point that is expected to improve the most upon the existing best observed point $y^* \geq \hat y(x_i) \forall i \in (1, \dots, t)$  where  $t$ is the number of observations thus far. Assume an improvement function $I(x)$ describing the improvement of the posterior of the GP over the best observed point:
\begin{equation}
    I(x) = \max(\tilde f_{\theta}(x) -y^*, 0)
\end{equation}
If there is no improvement, $I(x)=0$. For EI, the expectation of the improvement is:
\begin{equation}
    EI(x) = \mathbb E_{y}[I(x)]
\end{equation}
After some manipulations, the following closed form for the expected improvement is found:
\begin{equation}
    EI(x) =(\tilde \mu_{\theta}(x)-\hat y^*)\Phi(Z^*) + \tilde \sigma_{\theta}(x) \phi(Z^*)
\end{equation}
where $Z^*= \frac{y^*-\tilde\mu_{\theta}(x)}{\tilde \sigma_{\theta}(x)}$.  Then, solve the following optimisation problem to select the next experiment $x_{next}$.
\begin{equation}
    x_{\text{next}} = \text{argmax}_{x} EI(x)
\end{equation}
Since EI has a closed analytical expression for the expectation and derivative, gradient based second-order optimisation algorithms are typically used. 

While EI is a good baseline acquisition function, it can be overly exploitative, especially in the presence of noisy or inaccurate predictions from the GP \cite{Letham2019}.  Noisy EI (NEI) improves upon this by sampling the expectation of the acquisition function and the posterior probability, which reduces uncertainty in the optimum \cite{Letham2019}. A robust and efficient implementation of the NEI that uses a full Monte-Carlo treatment is available in the software package BoTorch \cite{Balandat2020}:
\begin{equation}
qNEI(x)= E[(\max \xi  - \max \xi_{obs} )_+]
\end{equation}
where $\xi$ represents samples from the posterior of the GP.

\subsection{Transfer learning for Bayesian optimisation}

The primary transfer learning approach in Bayesian optimisation involves enhancing the probabilistic model with auxiliary task information. I will review two methods for improving Gaussian processes: multi-task BO and multi-fidelity BO.

One of the potential ways to accelerate Bayesian optimisation is multi-task Bayesian optimisation (MTBO). MTBO was originally developed to speed up hyperparameter tuning of machine learning models (e.g., choosing learning rates and batch sizes to maximise model accuracy) \cite{Swersky2013}. Using only the hyperparameters and resulting model accuracy scores of a previously trained machine learning model (the source task),  MTBO decreased by up to 50\% the number of experiments needed to find optimal hyperparameters for a new machine learning model (the target task).  The data from the source task helped the model better predict target task, even with only a few experiments for the target task. 

One of the challenges faced in applying MTBO to big data uses cases has been its lack of scalability. This in mainly due to the $O(n^3)$ cost of using GPs with exact inference. There are several approaches to solving this problem including training a vanilla neural network on the tasks and feeding the output to a Bayesian linear regression model \cite{Perrone2018} and using the auxiliary tasks to create a learned feature representation in a compressed space \cite{Hakhamaneshi2021}. These methods have retained the performance of multi-task BO while minimising the computational cost of its deployment. However, in this thesis, all applications of MTBO have less than 1000 data points and, in many cases, less than 100. This makes multi-task GPs with exact inference more tractable.

Another direction has been to apply the machinery of multi-task GPs to multi-fidelity BO, which aims to leverage data from a cheap but less accurate data source to help with optimising a more expensive function \cite{Huang2006, Forrester2007}. This technique has been applied in a wide range of scenarios including parameter estimation of physics models using data from multiple experiments and simulations \cite{Perdikaris2016}; optimising battery electrode structure using data from simulations of varying speeds and accuracies \cite{Pan2017, Folch2023}, and optimising composition of alloys using a mix of DFT fidelities \cite{Tran2020}. I explore this idea in Chapter \ref{ch:mfbo}.

\section{Molecular Property Prediction}

In addition to Bayesian optimisation, I use machine learning for molecular property prediction in Part III. Molecular property prediction relies on supervised learning, an approach that aims to create models the predict one or more outputs given an input. In molecular property prediction, the input to a model is a parameterisation of a molecule, and the output is a property such as an activity coefficient.

\subsection{Molecular Parameterisation}

The first step in making machine learning models that can predict molecular properties is translating molecules to a machine-readable, typically numerical, format that can be used as input for ML models \cite{Wigh2022}. This translation process is molecular parameterisation as it aims to capture relevant molecular features for a particular property. For different chemical transformations, molecular parameterisations should capture different features such as steric hinderance of a functional group or electronegativity of neighboring atoms. The parameterisation strategy should also be chosen to allow optimal compatibility with the ML model used, as prediction performance will depend on the compatibility between input format and ML model.

Extended-connectivity fingerprints (ECFP) is a parameterisation method that captures atom types, neighboring connectivity relationships, bond types and represents the outcome in a machine-readable one-dimensional bit-vector \cite{Rogers2010}. Circular fingerprints (e.g. Morgan fingerprints) are generated by: 1) assigning identifiers to each atom in the molecule, 2) updating each atom’s identifiers depending on the neighboring atoms, 3) removing duplicates and 4) compressing the data into a vector of set length e.g. 1024 bit (a number of zeros and ones). One of the advantages of these fingerprint based methods is that they are considered cheap features for modelling; their generation does not require vast amount of computing power/time. Yet, their ability to explicitly capture molecular properties (e.g., sterics, electronics) of molecules is limited due to the lack of explicit modeling of charge density. Typically, models that use fingerprints correlate these effects implicitly through molecular functional groups associated such as electronegativity associated with different halides \cite{Eyke2020}.

An alternative approach is calculating molecular descriptors using density functional theory (DFT). DFT can be used to determine the ground/excited state of molecules and thus offer fundamental insights into geometric and electronic properties \cite{Shields2021}. As a result, DFT can be used to calculate descriptors that quantify the specific chemical properties of the given set of ligands such as the bulkiness of a molecule or electronegativity of atoms within a molecule. However, DFT calculations for large numbers of molecules are often more time-consuming than actually conducting the experiments for the relevant property \cite{Shields2021}.

More recently, parameterisation work has used neural networks to achieve the tailored nature of DFT descriptors without the computational expense. This work is divided into two approaches: natural language processing models and graph neural networks. The former leverages recent advances in language models such as transformers \cite{Vaswani2017}, where results can be achieved by training a model to predict the next word in a sentence across a wide variety of text. Since chemistry can be represented as a language in the form of simplified molecular-input line-entry system (SMILES) \cite{Weininger1988}, a language model can be trained to predict the next atom in a molecule when given only a portion of the molecule, thereby saving computational expense \cite{Schwaller2019}. Since the model must understand a significant volume of chemistry to be able to predict a SMILES string, its numerical output can be used as a “learned fingerprint” for other prediction tasks \cite{Schwaller2021, Pomberger2023}. Furthermore, the learned fingerprint can be tuned for each target task using standard neural network training. Alternatively, graph neural networks represent a molecule as an interconnected network of atoms and bonds \cite{Kearnes2016}. These networks can be trained to produce a “learned fingerprint” for prediction tasks \cite{Yang2019}. One of the most widely used forms of graph neural networks in chemistry are message passing neural networks \cite{Gilmer2017}, which are detailed in the following section.

An overview of the techniques covered in this section are shown in Table \ref{tab:parameterization}.

\begin{sidewaystable}
    \caption{Overview of the commonly used molecular parameterisation techniques for modelling chemical data.}
    \begin{tabular}{cp{0.25\linewidth}cc}
         Parameterisation
    Method & Information Captured & Data Type & Example Data  \\
        \hline
         OHE & Existence/absence of a molecule & Binary encoding & [0 0 0 1 0 0 0 ] \\
         Molecular fingerprints & Atom type, atom count, chemical structure, connectivity & Binary encoding & [1 0 0 1 1 0 1 0 0 … 0 1] \\
         DFT descriptors & Inter atomic information: length, angles, volumes &  Numerical values & 0.001342, 45, $\dots$ \\
         Learned representations & Connectivity and potentially atom and bond & Numerical values & 0.001342, 45, $\dots$
    \end{tabular}
    \label{tab:parameterization}
\end{sidewaystable}

\subsection{Deep Learning Architectures}

Molecules can be treated as graphs with atoms as nodes and bonds as edges. Therefore, message passing neural networks (MPNN) that operate on graphs can be used to create learned representations of molecules \cite{Gilmer2017}.  In a MPNN, each atom has associated node features $x_v$ such as atomic number and formal charge, and each bond has associated edge features $e_{vw}$ such as bond degree (single, double, triple). A hidden representation can be generated for each atom $h_t$, and  during a forward pass, messages are created as follows:
\begin{equation}
    m_v^{t+1} = \sum_{w\in N(v)} M_t(h_v^t, h_w^t, e_{vw})
\end{equation}

\begin{equation}
    h_v^{t+1} = U_t(h_v^t, m_v^{t+1})
\end{equation}
where $M_t$ is the message function, $U_t$ is the update function and $h_v^{t}$ is the hidden state at step $t$. This update process is repeated several times (usually two to three) and the representations from all atoms are aggregated, usually using a sum or mean:
\begin{equation}
    \bar h = \sum_{v\forall \mathcal V} h_v
\end{equation}

To obtain predictions $\hat y$, the outputs of the last message passing step $T$ are passed through a feed forward network $R$ in a readout phase:

\begin{equation}
    \hat y = R(h_v^T \in G)
\end{equation}

In contrast to traditional fixed fingerprints like ECFP \cite{Rogers2010}, the best feature vector is learned end-to-end for each property prediction task. I make use of MPNNs in Part III of this thesis.

\subsection{Transfer Learning for Molecular Property Prediction}

Most transfer learning work for molecular property prediction centers around pre-training. There is work on both self-supervised and supervised approaches.

The self-supervised approaches generally try to improve the representations learned by MPNNs using unlabeled data. Since there are large databases of existing molecules (e.g., ZINC with over one billion molecules \cite{Irwin2020}), a large-scale pretraining task similar to those used for language modeling can be constructed. For example, Hu et al. masked missing atom nodes from molecular graphs and created a task of predicting the missing nodes \cite{Hu2020Pretrain}. By combining this pre-training task with supervised pretraining on molecular properties, they obtained state-of-the-art results at the time of publication on some benchmark tasks. While self-supervised approaches are promising, they currently do not result in significant performance improvements without supervised pretraining \cite{Sun2022}.

Supervised approaches usually train a full network on previous data, fix the message passing encoder and only train the feed forward network on subsequent data. For example, Vermeire et al. used such an approach to improve predictions of solvation free energies by first pretraining on a large database of calculations from the thermodynamic programme COSMO-RS and subsequently fine-tuning on experimental data \cite{Vermeire2021, Vermeire2022}. They found that this combination of pretraining and fine-tuning outperformed using the COSMO-RS or experimental dataset individually. This approach has also been applied to predicting glass transition temperature of polymers \cite{Volgin2022}, fuel properties \cite{Larsson2023}, vapor pressure of energetic materials \cite{Lansford2023}, and several thermochemistry properties \cite{Grambow2019, Ureel2023}. I explore pretraining of MPNNs in Part III.

\section{Conclusions}

In this thesis, transfer learning is examined in two modes: multi-task learning, multi-fidelty learning and pretraining. These general strategies are overlaid on the machine learning approaches of Bayesian optimisation and molecular property prediction using deep learning. 