%************************************************
\chapter{ControlBO: Bayesian optimisation for Controller Tuning}\label{ch:mfbo} 
%************************************************

\section{Introduction}
\label{sec:intro}
Laboratory unit operations such as reactors and distillation columns  are often subjected to many processes in a short window of time. For each new process, the control systems of these unit operations need to be re-tuned, which can be a time and labor intensive endeavor. As an example, in the distillation laboratories at BASF, operators often spend days attempting to find the best tuning parameters for a new distillation column setup, which is unacceptable given the short timelines of operating a testing laboratory. While it is possible to use model predictive control, decentralised PID controllers are often used due to their robustness and simplicity. In this chapter, I aim to develop a fast method for tuning controllers and apply this method to laboratory distillation columns.

As noted in Chapter \ref{ch:rl_tuning}, there are a wide variety of existing tuning protocols ranging from heuristics such as Ziegler Nichols \cite{Ziegler1942} and its refinements \cite{Hang1991} to tuning using analytical equations from models via methods such as  Internal Model Control \cite{Copeland2010}. However, extending these to multi-input multi-output systems has been challenging, often requiring additional optimisation iterations \cite{Nandong2013, Nandong2015}. Optimisation-based methods have shown promise \cite{Pajares2019, Sumana2010, Rajapandiyan2012, Behroozsarand2012}, but these methods have been difficult to apply in practice due to the large number of iterations they require. Recently, Bayesian optimisation (BO) has shown promise for tuning controllers \cite{NeumannBrosig2020, Fiducioso2019, Khosravi2020, Konig2020, Fujimoto2022, Brunzema2022, Khosravi2022}, but this method still often requires tens to hundreds of iterations, making it too materially expensive for laboratory chemical processes.

One approach to reduce the number of iterations required for BO is multi-fidelity Bayesian optimisation (MFBO). MFBO leverages data from an auxiliary simulation or data source that is faster or cheaper to evaluate. Early examples of this approach include Kennedy et al. who used low fidelity approximations of an oil rig simulation to reduce the number of calls to a slower simulator \cite{Kennedy2000}. More recently, MFBO approaches have developed to accelerate the tuning of hyperparameters for machine learning models \cite{pmlr-v70-kandasamy17a}, design of batteries \cite{Folch2023} and optimisation of airplane rotors \cite{Pan2017}. Readers should refer to Chapter \ref{ch:background} for review of multi-fidelity and transfer learning in Bayesian optimisation.

In this chapter, I use Bayesian optimisation to accelerate controller tuning. I introduce a novel BO strategy called ControlBO that I  designed specifically for tuning control systems and explore whether multi-fidelity BO can accelerate controller tuning. I then develop a dynamic simulation of a distillation column that I use to evaluate ControlBO \textit{in silico}. This overall workflow is shown in Figure \ref{fig:tuning_workflow}.

\begin{figure}
    \centering
    \includegraphics[width=0.8\textwidth]{gfx/Chapter06/tuning_workflow.png}
    \caption{Three step workflow for distillation controller design and tuning.}
    \label{fig:tuning_workflow}
\end{figure}

\section{Methods}

\subsection{Controller tuning as an optimisation problem}

As noted in Chapter \ref{ch:rl_tuning}, controller tuning involves identifying the three parameters of the PID equation, $K^P$, $\tau_I$ and $\tau_D$:

\begin{equation}
    u(t) = PI(Y, Y^{sp}, \Omega) =  K^P e(t) + \frac{1}{\tau_I}\int_0^t e(t')dt' + \tau_D \frac{de(t)}{dt}
\end{equation}
\begin{equation}
    e(t) = (Y^{sp} - Y)
\end{equation}
where $u(t)$ is the controller output that is feed into the actuator (e.g., a valve) and $e(t)$ is the error at time $t$. $Y$ is the controlled variable, and $Y^{sp}$ is the setpoint, and $\Omega=\{K^p, \tau_I, \tau_D \}$ are the controller parameters. In this work, I only use PI controllers, so $\tau_D=0$.

I formulate the problem of controller design and tuning as an optimisation problem. The aim is to find a set of controller parameters that will achieve desired behavior. I define good controller behavior as minimising the integral absolute error (IAE) for each controlled variable and minimising total controller movement (CM):
\begin{equation}
    \min_{\Omega}(IAE_1, IAE_2, IAE_3, CM_1, CM_2, CM_3)
\end{equation}


\begin{equation}
    IAE = \int_0^T \vert y_{sp}(t) - \hat y(t) \vert dt
\end{equation}

\begin{equation}
    CM = \sum_{t=1}^{T}( \vert x_t - x_{t-1} \vert)
\end{equation}
where $y_{sp}$ is the setpoint,  $\hat y$  is the value of  the variable, T is the time horizon for a single tuning experiment. 

\subsection{ControlBO: Bayesian optimisation for controller tuning}

ControlBO is a Bayesian optimisation algorithm that aims to identify robust controller parameters for MIMO systems in as few experiments as possible. I was inspired by a recent paper by Paulson et al. that uses a Gaussian Process to model both the uncertain variables as well as the controller parameters \cite{Paulson2022}. They then use a min-max formulation where the integral absolute error is minimised, while the uncertain variables are held at their maximum value. However, they sample from a high fidelity simulation and then assume that this simulation will always transfer to laboratory experiments. In contrast, I use both simulation and experimental data so that the optimisation algorithm can find optimal tuning parameters even if the simulation is not accurate.

ControlBO is a multi-fidelity and multi-objective Bayesian optimisation algorithm that aims to solve the following problem:

\begin{equation}
    \Omega^* = \min_{\Omega} \mathbf f(\Omega, \delta)
\end{equation}
where $\Omega$ are the controller parameters and $\mathbf f$ is the laboratory controller tuning problem. $\delta$ are disturbance variables that might be stepped by operators or changed by other external factors. Similar to Paulson et al. \cite{Paulson2022}, I use a min max formulation where a GP is trained on both controller parameters $\Omega$ and values of disturbance variables from past tuning experiments. During acquisition function optimisation, the disturbance variables are fixed at their maximum value, while the controllers parameters are varied to maximise the acquisition function (i.e., minimise IAE and CM). Fixing the disturbance variables ensures that robust controller parameters are chosen.

The unique feature of ControlBO is that it trains an independent multitask GP to predict each controller objective (IAE) and controller movement (CM) given the relevant controller parameters. The multitask GP is trained on both auxiliary simulation data and tuning experiments. By using data from the auxiliary simulation, my hope is that the optimisation will proceed faster than when using only data from tuning experiments executed on the distillation column since the multi-task GP could learn the correlation between the simulation and the experimental data and thus make better predictions on the experiments with less data.

Since the problem is multi-objective, I used a multi-objective acquisition function, namely the q Expected Hypervolume Improvement (qEHVI) \cite{Balandat2020}. This acquisition function has a balance of computational efficiency and fidelity resulting in an ability to find the Pareto front of trade-offs between objectives quickly. 

Additionally, I found that tuning experiments often fail due to controller parameters being too aggressive. ControlBO therefore trains a GP classifier that predicts the probability of success of a tuning experiment using data from the auxiliary simulation. Success is defined as whether or not a simulation converged. During acquisition function optimisation, the probability of success of a tuning experiment, as predicted by the GP classifier, is multiplied by the acquisition function value to determine a final acquisition function value that is optimised to choose the next experiment:

\begin{equation}
    \alpha_{c} = p_{success}(y_c \vert x, \mathcal D) \alpha_{qEHVI}(X)
\end{equation}

where $\alpha$ is the acquisition function. By maximising $\alpha_c$, the likelihood of finding controller parameters that both improve performance and are safe to execute should be improved. This is a similar concept to contemporaneous work on meta-learning safe priors for controller tuning \cite{Rothfuss2023}. 

% Additionally, I found that the simulations often failed with certain controller parameters. Therefore, I trained a classifier to predict the likelihood of a successful simulation, which I expect to also correlate with reasonable experimental parameters. The probability of success predicted by the classifier was multiplied by the qNEHVI acquisition function value to form a chance constrained acquisition function:

% \begin{equation}
%     \alpha_{cc} = p(h \vert \mathcal D) \alpha_{qNEHVI}
% \end{equation} 

I developed a differential equation simulation of a distillation column that was aligned to experimental data using both steady state and dynamic parameter estimation. I generated two versions of this simulation: a "high fidelity" version that was aligned to experimental data and a "low fidelity" copy with slightly adjusted parameters. For evaluation of ControlBO, I first generated the data from a low fidelity simulation in advance with 500 randomly generated tuning experiments run in simulation. Then, I used high fidelity simulation to represent the actual laboratory distillation column for evaluation of ControlBO \textit{in silico}. In the following section, I describe the simulation in detail.

\subsection{Distillation Simulation}\label{sec:distillation_model}

In order to base the distillation simulations in reality, my collaborators at BASF collected experimental data for separation of a 50/50 methanol-water mixture at the laboratories of BASF SE in a 80 tray column with one bubble cap on each stage. The column was kept at approximately 800 mbar vacuum. The column occupies approximately three stories in the BASF lab.  

% Three days of data were collected, one of which is used for parameter estimation and the rest for validation. 

My goal in simulating distillation columns was to find a balance between model complexity and accuracy. I opted for a formulation similar to \citet{Diehl2001}, which assumes constant pressure drop over time but includes the MESH equations and hydrodynamics. As shown in Figure \ref{fig:column}, $N$ is the number of trays in the distillation column. The distillation column is numbered from the bottom, starting with the reboiler as $i=0$ and condenser as $i=N+1$.  

\begin{figure}
    \centering
    \includegraphics{gfx/Chapter06/basic_column.png}
    \caption{Depiction of a distillation column with $N$ trays, feed at stage $N_F$, and reboiler and condenser as stage 0 and $N+1$ respectively.}
    \label{fig:column}
\end{figure}


% The following are parameters of the model, determined by the experimental scenario or estimated via parameter estimation (see the next section).

% \begin{itemize}
%     \item$P_{top}$ is the pressure above stage $N$ in the column
%     \item$\Delta P_{strip}$ is the pressure drop in the stripping section (below feed stage)
%     \item$\Delta P_{rect}$ is the pressure drop in the rectifying section (feed stage and above)
%     \item$q_{N_F}$ is the vapor fraction in the feed
%     \item$z_{N_F}$ is the mol fraction in the feed
%     \item$F$ is the feed flowrate (kmol/hr)
%     \item$Q_{reb}$ is reboiler duty
%     \item$Q_{cond}$ is the condenser duty
%     \item$\alpha_{strip}$ is the Murphree efficiency in the stripping section (below feed stage)
%     \item$\alpha_{rect}$ is the Murphree efficiency in the rectifying section (feed stage and above)
% \end{itemize}

I assumed the pressure drop was constant throughout the simulation, which is reasonable given the experimental observations (a vacuum was used to keep pressure drop constant). Therefore the pressure on each stage is calculated using:

% \begin{table}
%     \centering
%     \caption{Parameters of the distillation column model determined by experiments.}
%     \begin{tabular}{cccc}
%         \textbf{Parameter} & \textbf{Description} & \textbf{Scenario 1} & \textbf{Scenario 2}  \\
%         \hline
%          $P_{top}$ &  Pressure on stage N of the column &  0.5 & 0.22 \\ 
%          $q_{N_F}$ & vapor fraction  in the feed & 1.0 & 1.0 \\
%          $z_{N_F}$  & mole fraction of methanol in the feed &  0.5  & 0.5 \\
%          $F$ & Feed flow rate &  333 mol/min  & 332 mol/min \\
%          \hline
%     \end{tabular}
%     \label{tab:parameters}
% \end{table}

\begin{equation}
    P_i = P_{i+1} + \Delta P_i \;\; i=0\dots N
\end{equation}

where $\Delta P_i =\Delta P_{strip}$ for stages below feed stage and $\Delta P_i =\Delta P_{rect}$ for feed stage and above. The condenser pressure is calculated using $P_{cond}=P_{N} - \Delta P_{rect}$. State variables  are shown in Table \ref{tab:state_variables}.
\begin{table}
    \centering
    \caption{State variables of the distillation column model.}
    \begin{tabular}{cc}
        \textbf{Variable} & \textbf{Description}  \\
        \hline
         $x_i$ &  Liquid composition of methanol on stage $i$ \\
         $y_i$ & Vapor composition of water on stage $i$\\
         $L_i$ & Liquid flow leaving stage (mol/minute) $i$  \\
         $V_i$  & Vapor flow leaving stage (mol/minute) $i$ \\
         $T_i$  & Temperature on stage (K) $i$ \\
         $n_i$  & Holdup on stage (mol) $i$ \\
         \hline
    \end{tabular}
    \label{tab:state_variables}
\end{table}

\subsubsection{Material balances}
Material balances are calculated for all stages as follows:

\begin{equation}
\frac{dn_i}{dt} = L_{i+1}-L_i + V_{i-1}-V_i \;\; \forall i=1 \dots N_F-1, N_F+2 \dots N+1
\end{equation}

\noindent Note that the feed stage and the stage above the feed stage are treated separately:

\begin{equation}
    \frac{dn_{N_F}}{dt} = L_{N_F+1}-L_{N_F} + V_{N_F-1}-V_{N_F} + q_{N_F}F   
\end{equation}

\begin{equation}
   \frac{dn_{N_F+1}}{dt} = L_{N_F+2}-L_{N_F+1} + V_{N_F}-V_{N_F+1} + (1-q_{N_F})F 
\end{equation}
where $q_i$ is the liquid fraction of the feed stream and $F$ is the molar feed flow. The reboiler and sump are lumped into into one stage. 

\begin{equation}
    \frac{dn_1}{dt} =  L_1 - V_0 - B
\end{equation}
$B$ is the molar bottoms flow. The condenser and reflux drum are lumped and consider as one unit.

\begin{equation}
    \frac{dn_{N+1}}{dt} = V_{N+1}-(L_{N+1} + D)    
\end{equation}


\subsubsection{Component balances}

Component balances are calculated for all trays:
\begin{equation}
   \frac{d(x_{i} n_{i})}{dt}= L_{i+1} x_{i+1} - L_i x_{i} + V_{i-1} y_{i-1} -V_i y_i  \;\; \forall i=1 \dots N_F-1, N_F+2 \dots N+1
\end{equation}

\noindent The feed stage and the stage above the feed stage are treated separately:

\begin{equation}
\begin{split}
 \frac{d(x_{N_F} n_{N_F})}{dt} =  &  x_{N_F+1}L_{N_F+1}  - x_{N_F}L_{N_F} +  y_{N_F-1}V_{N_F-1} \\ & -y_{N_F}V_{N_F} + q_{N_F}z_FF   
\end{split}
\end{equation}
where $z_F$ is the overall molar composition of the feed stream.

\begin{equation}
\begin{split}
    \frac{d(x_{N_F+1} n_{N_F+1})}{dt} =  & x_{N_F}L_{N_F} - x_{N_F+1}L_{N_F+1} +  y_{N_F+2}V_{N_F+2}   \\ & -y_{N_F+1}V_{N_F+1}  + (1-q_{N_F})z_FF 
\end{split}
\end{equation}

\noindent The component balance around the sump is as follows:

\begin{equation}
    \frac{d(x_{0} n_{0})}{dt} = L_1 x_1 - (B + V_0) x_0 
\end{equation}

\noindent The component balance around the condenser is as follows:
\begin{equation}
    \frac{d(x_{N+1} n_{N+1})}{dt}  = V_{N+1} y_{N+1} - x_{N+1} (L_{N+1} + D)
\end{equation}

\subsubsection{Energy balances}\label{sec:energy_balances}

In calculating energy balances, I assume that vapor holdup is negligible and, therefore, only consider the liquid holdup. Energy balances are calculated using the liquid and vapor enthalpy, as detailed subsequently. For the trays:

\begin{equation}
\begin{split}
    \frac{d(n_ih^L_i)}{dt} = h^L_{i+1}L_{i+1}-h^L_iL_i+h^V_{i-1}V_{i-1}-h^V_iV_i \\ \forall i=1 \dots N_F-1, N_F+2 \dots N+1
\end{split}
\end{equation}
where $h^L_i$ and $h^V_i$ are the specific enthalpy of the liquid and vapor respectively on stage $i$. For the feed stage and the stage above the feed stage:

\begin{equation}
\begin{split}
    \frac{d(n_{N_F}h^L_{N_F})}{dt} = & h^L_{N_F+1}L_{N_F+1}-h^L_{N_F}L_{N_F} + h^V_{N_F-1}V_{N_F-1}-h^V_{N_F}V_{N_F} \\ &  + q_{N_F}F h^L_{N_F}
\end{split}
\end{equation}


\begin{equation}
\begin{split}
    \frac{d(n_{N_F+1}h^L_{N_F+1})}{dt} = & h^L_{N_F+2}L_{N_F+2}-h^L_{N_F+1}L_{N_F+1} +h^V_{N_F} V_{N_F}-\\ &h^V_{N_F+1} V_{N_F+1} + (1-q_{N_F})Fh^V_{N_F}
\end{split}
\end{equation}

 The energy balances for the sump is:

\begin{equation}
    \frac{d(n_0h^L_0)}{dt} = L_1 h^L_1 - (V_0 + B) h^L_0 + Q_0
\end{equation}

where $Q_0$ is the reboiler heat duty.

\noindent The energy balance for the condenser is as follows:

\begin{equation}
    \frac{d(n_{N+1}h^L_{N+1})}{dt} = h^V_{N+1} V_{N+1}-h^L_{N+1}(L_{N+1} + D) + Q_{N+1}
\end{equation}
where $Q_{N+1}$ is the condenser duty. Additionally, a heat exchanger subcools the liquid from the condenser:

\begin{equation}
    n_{subcool}\frac{d h^L_{reflux}}{dt} = L_{N+1}(h^L_{N+1} - h^L_{reflux}) + Q_{subcool}
\end{equation}
where $Q_{subcool}$ is the heat exchanger duty. For liquid enthalpy $h^{L,c}$ of each component, a mole fraction weighted average of the specific enthalpies is calculated by integrating a constant heat capacity from a reference temperature $T_{ref}$ of normal boiling point of each component $c$. Data is taken from Table 2-72 of Perry’s Chemical Engineer's Handbook \cite{Perrys2018}.

\begin{equation}
    h^L(T_i, P_i, x_i) = x_ih^{L,0}(T_i,P_i) + (1-x_i)h^{L,1}(T_i,P_i)
\end{equation}

\begin{equation}
    h^{L,c}(T,P) = \int_{T_{ref}}^T C_p^c(\tau)d\tau 
\end{equation}

\begin{equation}
    C_p^c(T) = \sum_{k=0}^5 c^kT^k 
\end{equation}

\noindent The enthalpy of the liquid on stage $i$ is then:
\begin{equation}
    h^L_i := h^L(T_i, P_i,x_i) \;\; \forall i = 1 \dots N_F-1, N_F+1 \dots N+1
\end{equation}

\begin{equation}
    h^L_{N_F} := h^L(T_{N_F}, P_{N_F},x_{N_F})
\end{equation}

\noindent For vapor enthalpy,  a similar approach is taken. Data for heat of vaporisation, vapor enthalpy and heat capacity are taken from Table 2-169, Table 2-102, Table 2-27 respectively in Perry’s Chemical Engineer's Handbook \cite{Perrys2018}.

\begin{equation}
    h^V(T_i, P_i, x_i) = x_i h^{V,0}(T_i,P_i) + (1-x_i) h^{V,1}(T_i,P_i)
\end{equation}

\begin{equation}
    h^{V,c}(T,P) = \int_{T_{ref}}^T C_{p,V}^c(\tau)d\tau 
\end{equation}

\begin{equation}
    C_{p,V}^c(T) = \sum_{k=0}^5 c_v^kT^k 
\end{equation}

\begin{equation}
    h^V_i := h^V(T_i, P_i,x_i)
\end{equation}

\begin{equation}
    h^V_{N_F} := h^V(T_{N_F}, P_{N_F},x_{N_F})
\end{equation}


\subsubsection{Activity coefficient model}

I use the NRTL model for calculating the activity coefficients, though a variety of models could be used. The NRTL parameters for methanol-water are shown in Table \ref{tab:nrtl_parameters}.

\begin{table}[]
    \centering
    \caption{Methanol-water NRTL parameters. Taken from Beneke et al.\cite{Beneke2012}.}
    % https://onlinelibrary.wiley.com/doi/pdf/10.1002/9781118477304.app2
    \begin{tabular}{cc}
        Parameter & Value \\
        \hline
         $\tau_{12}$ & -0.693  \\
         $\tau_{21}$ & 2.732   \\
         $\alpha_{12}$ & 0.3 \\
    \end{tabular}
    \label{tab:nrtl_parameters}
\end{table}

\begin{equation}
    \ln \gamma_{1} = x_2^2\biggl [\tau_{21} \biggl (\frac{G_{21}}{x_1+x_2G_{21}}\biggr)^2 + \frac{\tau_{12}G_{12}}{(x_2 + x_1 G_{12})^2} \biggr ]
\end{equation}

\begin{equation}
     \ln \gamma_{2} = x_1^2\biggl [\tau_{12} \biggl (\frac{G_{12}}{x_2+x_1G_{12}}\biggr)^2 + \frac{\tau_{21}G_{21}}{(x_1 + x_2 G_{21})^2} \biggr ]   
\end{equation}

\begin{equation}
    G_{ij} = e^{-\alpha_{ij} \tau_{ij}}
\end{equation}

\subsubsection{Pressure balance}

Vapor pressure is calculated using Antoine’s equation with coefficients taken from NIST. 

\begin{equation}
    \log P^{sat}(T_i) = A + \frac{B}{T_i + C}
\end{equation}

For trays ($i=1\dots N$), modified Raoult’s Law is used with Murphree efficiencies $\alpha$ to account for deviations from equilibrium.

\begin{equation}
    y_i = \alpha_i\frac{x_i\gamma_i(T_i, P_i, x_i)P_0^{sat}(T_i)}{P} + (1-\alpha_i)y_{i-1}
\end{equation}

This leads to the bubble pressure balance:

\begin{equation}
    P = x_i\gamma_{i,1}(T_i, P_i, x_i)P^{sat}_{i,0} + (1-x_i)\gamma_{i,2}(T_i, P_i, (1-x_i))P^{sat}_{i,1}
\end{equation}

\subsubsection{Francis weir formula}\label{sec:francis_weir}

The Francis weir formula is used for dynamic simulations to calculate liquid flow over the weir given a certain holdup. As shown in Figure \ref{fig:weir}, $h_i^{ow}$ from the clear liquid to the weir. This can be calculated by taking the difference between the actual volume in the weir and a reference volume $n^v_{ref}$ (i.e., the volume if there was no crest) given the cross sectional area $A$.

\begin{figure}
    \centering
    \includegraphics[width=0.4\textwidth]{gfx/Chapter06/weir.png}
    \caption{Diagram of weir showing $h_{ow}$ measured from the top of the clear liquid on a tray.}
    \label{fig:weir}
\end{figure}


\begin{equation}
    h^{ow}_i = \frac{n^v_i-n^v_{ref}}{A}
\end{equation}

\noindent The Francis weir formula for calculating flow over the weir of width $w$ comes from the Bernoulli equation for the horizontal velocity $v$. The equation is integrated from the top of the liquid (y=0) to the weir ($y=h_i^{ow}$)
\begin{equation}
    \frac{1}{2} \rho v^2 = \rho g y \ \Longrightarrow v(y) = \sqrt{2gy}
\end{equation}
\begin{equation}
    L_i^{vol} = \int_0^{h_i^{ow}} v(h)wdh = w\sqrt{2g}\int_0^{h_i^{ow}} h^{1/2}dh = \frac{2}{3}w\sqrt{2g}(h_i^{ow})^{3/2} 
\end{equation}
Substituting for $h_i^{ow}$:
\begin{equation}
   L_i^{vol} = \frac{2}{3} w\sqrt{2g}\biggl(\frac{n^v_i-n^v_{ref}}{A}\biggr)^{3/2}  
\end{equation}

\noindent All constants are then lumped into a a single parameter $W_{tray}$:
\begin{equation}
    L_i^{vol} = W_{tray}(n_i^v-n^v_{ref})^{3/2}
\end{equation}

\begin{equation}
    W_{tray} = \frac{2}{3}w\sqrt{2g}A^{-3/2}
\end{equation}

Based on measurements from BASF, I assume the diameter of the weir $w$ is 8 mm and the height of the weir $h$ is 10 mm. Additionally, these measurements give $h = 1\cdot10^{-2} m$ and $A=\pi w^2/4 = \pi(8 \cdot 10^{-3})^2/4 = 5.024 \cdot 10^{-5} m^2$. Therefore $W_{tray}^{init}=120149 \; m^{-3/2}s^{-1}$. $n^v_{ref}=Ah=5.024 \cdot 10^{-7} m^3$. 

Since the simulation is in molar flowrates, volumetric flowrates should be converted to molar flowrates:
\begin{equation}
    L_i = \rho(T_i, x_i)L^{vol}_i  
\end{equation}
where $\rho(T_i,x_i)$ is the volumetric density, calculated by a weighted average of the volumetric density of both components. Volumetric density come from Perry’s Chemical Engineer's Handbook \cite{Perrys2018}.

\begin{equation}
    \rho(T_i, x_i) = x_i\rho^0(T_i, x_i) + (1-x_i)\rho^1(T_i, x_i)
\end{equation}

\subsubsection{Feedback control}

Three PI controllers are used to control the distillation column. The first two controllers are  level controllers:
\begin{equation}
    D := PI(l_{condenser}, \Omega_{cond})
\end{equation}
\begin{equation}
    B := PI(l_{sump}, \Omega_{sump})
\end{equation}
where $l$ is the level in the respective vessel. The level (\%) is calculated using the following linear equation which is based on data from BASF:
\begin{equation}
    l = c^1 n + c^2
\end{equation}
where $c^1=143.94 m^{-1}$ and $c^2=0.21$. A composition controller uses the temperature on stage 17 as the controlled variable and the reboiler heat as the manipulated variable.
\begin{equation}
    Q_0 := PI(T_{17}, \Omega_{composition})
\end{equation}

\subsection{Solving and aligning model using GEKKO}

I implemented the equations from the previous section in a simulation of a distillation column that can be aligned to experiments using a small amount of steady state data. The distillation simulation was built in Python using GEKKO \cite{Beal2018}, and the differential algebraic equation model used for the simulations is fully detailed in Section \ref{sec:distillation_model}. To align the simulation to experiments, a three step procedure, illustrated in Figure \ref{fig:simulation_workflow}, was followed: initialisation, steady state solution with parameter estimation and dynamic simulation. All simulations were solved with the APOPT nonlinear programming solver to a tolerance $10^{-9}$ and relative tolerance $10^{-9}$.

\begin{figure}
    \centering
    \includegraphics[width=\textwidth]{gfx/Chapter06/simulation_workflow.png}
    \caption{Schematic of the workflow for aligning the distillation simulation to experiments.}
    \label{fig:simulation_workflow}
\end{figure}

To initialise  the simulation, only the component material balances and pressure balances were solved using the steady-state solver in GEKKO \cite{Beal2018}. This solver turns off the derivatives in all equations and solves a nonlinear simultaneous equation problem using a nonlinear solver. The constant molal overflow assumption was used, which states that there is constant liquid flow in the stripping and rectifying section respectively as well as constant vapor flow.  The values of the state variables are saved from the the constant molal overflow calculation, and estimates of reboiler and condenser duty are calculated using energy balances.

The full steady state simulation was then solved using the state variables from the constant molal overflow simulation as initial guesses. Again the steady-state solver in  GEKKO was used. The results of the full steady state simulation was used to generate the nominal values for steady state parameter estimation to the experimental data. The full list of estimated parameters can be found in Table \ref{tab:param_estimation}. Note that, while the condenser duty during dynamic simulation was fixed at the value from steady state parameter estimation, the reboiler duty was a manipulated variable; therefore, the steady-state estimated value of the reboiler duty served as an initial condition. The objective of the parameter estimation was to minimise the $L_2$ error between the simulation and experimental temperatures on stages 1, 8, 11, 16, 17, 21, 23, 41, 47, 53, 61, and 81, as these stages were shown to be sensitive to changes in the parameters. Therefore, the parameter estimation problem was formulated as:
%  The $l_1$ norm was chosen due to its improved  handling of outliers \cite{Safdarnejad2016}.
\begin{subequations}
    \begin{align}
        \min_{\boldsymbol \theta_{est}} \sum_{i \in \Gamma} \lVert T_i-T_i^{exp}\rVert_2 + w_p \lVert \Delta P^{exp} - \Delta P \rVert_2 \\
        s.t. \;\; 0 = \mathbf g(\mathbf X, \boldsymbol \theta)
    \end{align}
\end{subequations}
where $\mathbf X$ are the state variables and $\boldsymbol \theta = \{\boldsymbol \theta_{est}, \boldsymbol \theta_{const} \}$.  $\boldsymbol \theta_{est}$ are the parameters to be estimated in Table \ref{tab:param_estimation} and $\boldsymbol \theta_{const}$ are set using data from geometry of the column. $\Delta P$ is the column pressure drop, and $w_p$ is a weighting for pressure drop objective, set nominally to 100.  $\mathbf f$ is the system of equations represented by equations in section \ref{sec:distillation_model} with all derivatives set equal to zero. 
% \{\mathbf x, \mathbf y, \mathbf T, \mathbf L, \mathbf n\}$ 
\begin{sidewaystable}[]
    \centering
    \caption{Parameters of the distillation column model estimated using experimental data. Estimates of the reboiler and condenser duties were made using the nominal values from the steady-state simulation.}
    \begin{tabular}{cccc}
        \textbf{Parameter} & \textbf{Description} & \textbf{Initial Guess} & \textbf{Estimated Value}  \\
        \hline
         $\alpha_{strip}$ & Murphree efficiency in stripping section &  0.5 & 0.48 \\ 
         $\alpha_{rect}$ & Murphree efficiency in rectifying section & 0.5 & 0.1 \\
         $\Delta P_{rect}$ & Pressure drop in rectifying section &  $\Delta P_{tot}/N$ & 0.96 mbar \\ 
         $\Delta P_{strip}$ & Pressure drop in stripping section &  $\Delta P_{tot}/N$ & 0.96 mbar \\ 
         $ Q_{reb}$ & Reboiler duty at steady state &  0.25 kW  & 0.267 kW  \\
         $ T_{cond}$ & Condenser temperature at steady state &  333 K  & 333 K\\
         $ Q_{subcool}$ & Subcooling heat exchanger duty & -0.25  kW & -0.41 kW \\
         \hline
    \end{tabular}
    \label{tab:param_estimation}
\end{sidewaystable}

I used the state variables after parameter estimation from steady state simulations as the initial condition for the dynamic simulation. The dynamic simulation was run using GEKKO's sequential differential algebraic equation solver. The simulation was stepped forward in thirty second intervals using a sequential integration approach. 

% In Figure \ref{fig:validation}a, we show that using the estimated parameters with settings from a different day of experiments gives good alignment between simulation and experiment.  An example closed-loop dynamic simulation is shown in Figure Xb, which demonstrates that the dynamic simulation aligns closely to the experimental dynamics. 

% \begin{figure}
%     \subfloat[]{
%         \centering
%         \includegraphics[clip,width=\textwidth]{gfx/Chapter06/2021_11_18_steady_state_temperature.png}
%     }

%     \subfloat[]{
%         \centering
%         \includegraphics[clip,width=\textwidth]{gfx/Chapter06/2021_11_18_steady_state_temperature.png}
%         \subcaption{}
%     }
%     \caption{Caption}
%     \label{fig:validation}
% \end{figure}

\section{Results}

\subsection{Distillation simulation correlates with experiments}

In order to build a simulation that could be used for examining the effectiveness of multi-fidelity BO for controller tuning, I used parameter estimation to align the simulation to one steady state experimental condition collected at BASF. The resulting steady-state simulation after parameter estimation is shown in Figure \ref{fig:estimated}. I used an average of the experimental stage temperatures over nine hours of steady state experimental data for parameter estimation. The final estimated parameters are shown in Table \ref{tab:param_estimation}.

The simulated steady state temperature profile aligns well with the experimentally measured temperatures (see Figure \ref{fig:estimated}). In particular, the simulation is able to capture the strong change in temperature between stage 10 and 20. However, the temperature of the middle section of the column is significantly lower than the experimentally measured values. This discrepancy could be due to a number of missing phenomena in the simulation including the heaters in the real column used to prevent heat loss from the column. These heaters could potentially increase the temperature of the internals of the column in a nonlinear fashion throughout the column. 

% Figure 3: Steady state simulation
\begin{figure}
    \centering
    \includegraphics[width=\textwidth]{gfx/Chapter06/2021_11_17_steady_state_estimated.png}
    \caption{Steady state simulation after parameter estimation of a separation of 50/50 methanol-water separation in a distillation column with 80 bubble cap trays. \textbf{Top left}: Composition profile of the column with liquid composition in solid lines and vapor composition in dotted lines. \textbf{Top right}: Temperature profile of the simulation and experiments. \textbf{Bottom left}: Flow profile in column. \textbf{Bottom right}: Pressure profile in column.}
    \label{fig:estimated}
\end{figure}

While the steady state simulation aligned well to experiments, I found that the dynamic simulation did not fully capture all aspects of the experiments as shown in Figure \ref{fig:dynamic_nominal}. In particular, the frequency response of the simulation upon a step in reflux rate did not reflect the oscillations seen in the actual experiments. Achieving a perfect match between the experimentally measured values and simulation is difficult, especially when the experiments are measured in closed-loop. However, I proceeded forward with Bayesian optimisation given that the simulations captured some aspects of the steady-state and dynamic behavior of the distillation column.
 
% Figure 4: Dynamic simulation
\begin{figure}
    \centering
    \includegraphics[width=\textwidth]{gfx/Chapter06/2021_11_17_closed_loop_dynamic.png}
    \caption{Dynamic simulation after parameter estimation of a separation of 50/50 methanol-water separation in a distillation column with 80 bubble cap trays. \textbf{Top left}: Liquid composition over time. \textbf{Top right}: Temperature over time. Thick lines represent experimental data. \textbf{Bottom left}: Flow over time \textbf{Bottom right}: Holdups in condenser and reboiler.}
    \label{fig:dynamic_nominal}
\end{figure}

\subsection{Optimisation of controller parameters}

To evaluate the performance of ControlBO, I analysed the performance of ControlBO with and without a multi-task model.  For the variations with a multi-task model, we used data from  500 auxiliary tuning experiments conducted on a simulation with a different Francis weir parameters than those shown in Section \ref{sec:francis_weir}. This was to simulate a case where we conducted parameter estimation on a single experimental condition, but there was still error between the simulation and experiments. In other words, I desired to see if the multi-task model could learn from an auxiliary simulation that was not a perfect reflection of the real experiment.  

% Figure 5: Hypervolume vs iterations
\begin{figure}
    \centering
    \includegraphics[width=\textwidth]{gfx/Chapter06/hypervolume_comparison.png}
    \caption{Hypervolume trajectory of a single optimisation run. Hypervolume is a measure of the volume dominated by the Pareto front; larger hypervolumes correspond with more optimal solutions discovered.}
    \label{fig:hypervolume_comparison}
\end{figure}

Figure \ref{fig:hypervolume_comparison} shows the hypervolume with respect to a fixed reference point during the tuning experiments. As noted in Chapter \ref{ch:summit}, hypervolume is a measure of the volume dominated by the Pareto front; larger hypervolumes correspond with more optimal solutions discovered (i.e., minimal IAE and CM). Note that the disturbance variables were the feed rate and reflux rate, which were stepped randomly during each tuning experiment. This reflects the actual use case where operators might perform steps on various streams and the tuning system needs to be able to accommodate these external changes.  All ControlBO variations were able to find better controller parameters over the course of 20 tuning experiments, improving their approximation of the Pareto front. Furthermore, the combination of a classifier and using auxiliary data from simulations via the multi-task model resulted in faster optimisation with optimal tuning parameters being found within 2-3 tuning experiments compared 5-10 with the single-task model.  Note that all tuning experiments were conducted in simulation, though the long-term intention is to use the algorithm directly on a laboratory distillation column.

In Figure \ref{fig:comparison_controlbo}, I create a performance objective to compare the final tuning of each optimisation strategy: 

\begin{equation}
\begin{split}
    \text{Overall Objective}= w_1 IAE_{sump} + w_2 IAE_{condenser} +  & \\ w_3 IAE_{composition} +  w_4 CM_{sump} + & \\  w_5 CM_{condenser} + w_6 CM_{composition}
\end{split}
\end{equation}
where $w_i$ is a weighting which is the inverse of the maximum bound of each objective (IAE or CM). The weighting is determined as follows:
\begin{equation}
    \frac{1}{w} = s1.5 * Y^{sp} \frac{T}{\Delta t}
\end{equation}
where $Y^{sp}$ is the setpoint value for the related controller, $T$ is the total time of the tuning experiment and $\Delta t$ is the timestep. This performance objective represents the overall performance of all controllers. ControlBO with a multi-task model and classifier performs best for both steps in reflux rate and feed rate. The multi-task model alone does not improve performance over the single-task model always. Overall, ControlBO performs well in selecting robust tuning parameters quickly.


\begin{figure}
    \centering
    \includegraphics[width=\textwidth]{gfx/Chapter06/comparison.png}
    \caption{Comparison of ControlBO using various combinations of multitask model and success classifier.}
    \label{fig:comparison_controlbo}
\end{figure}

% Subsequently, I tested whether a multitask model trained on 100 tuning experiments from an auxiliary simulation could improve the speed at which tuning occcurs. The multitask model should have better correlation but it doesn't. The success classifier doesn't help at all. This is likely due to the fact that the model is able to optimise the controller parameters without many changes and can find the best attainable parameters in  a few iterations. In other words, the problem is made easier by the poor quality of the distillation model.

% Controller plots before and after tuning

\subsection{How does the multi-task model perform?}

As shown in Table \ref{tab:mt_vs_st_controlbo}, the multi-task models are more accurate than the single-task models when predicting data held-out via random train-test split. Therefore, these models can be used to quickly find optimal controller parameters and guide optimisation. In particular, to produce the statistics shown in Table \ref{tab:mt_vs_st_controlbo}, the models were trained on the five initial random tuning experiments generated by Latin hypercube sampling in each tuning run, and the predictions were on the subsequent ten data points from one run. Therefore, these results suggest that the multitask model could more effectively select high quality experiments.
% Table with MAE and $\rho$when using single-task vs multitask model

\begin{table}[]
    \centering
    \caption{Comparison of mean absolute error (MAE) and Pearson's correlation coefficient ($\rho$) of predictions on five held out data points when model is trained on five data points selected randomly. The objectives are integral absolute error (IAE) and controller movement (CM). TC stands for temperature controller, and LC stands for level controller.  ST means single-task and MT means for multi-task.}
    \begin{tabular}{llrr}
    
    \toprule
                     &  &      MT &       ST \\
    Objective & {} &         &          \\
    \midrule
    Composition TC IAE & MAE&  \textbf{212.72} &  2094.70 \\
                     & $\rho$&    \textbf{0.99 }&     0.27 \\
    Composition TC Movement & MAE&    \textbf{0.51} &    11.04 \\
                     & $\rho$&    \textbf{1.00} &     0.53 \\
    Condenser LC IAE & MAE&    \textbf{0.00} &     0.30 \\
                     & $\rho$&    \textbf{0.28} &     0.03 \\
    Condenser LC Movement & MAE&    \textbf{0.25} &     0.45 \\
                     & $\rho$&    \textbf{0.39} &     0.30 \\
    Sump LC IAE & MAE&    \textbf{0.69} &     1.68 \\
                     & $\rho$&    \textbf{0.98} &     0.61 \\
    Sump LC Movement & MAE&    \textbf{0.10} &     0.42 \\
                     & $\rho$&    0.90 &    \textbf{ 0.96} \\
    \bottomrule
    \end{tabular}
    \label{tab:mt_vs_st_controlbo}
\end{table}


\subsection{Why does using a classifier improve performance?}

The classifier is an important component of ControlBO since it enables making maximal use of simulation data. As shown in Figure \ref{fig:classifier_calibration}, the classifier was well calibrated in that increases in predicted probability of success were correlated with high fractions of actual successful simulations, where success is defined as the simulation converging. This calibration is important since the classifier probabilities are used to weight the acquisition function values.

% Classifier calibration
\begin{figure}
    \centering
    \includegraphics[width=\textwidth]{gfx/Chapter06/classifier_calibration.png}
    \caption{Calibration curve of GP classifier trained on 80\% of auxiliary simulation data for ControlBO optimisation and tested on the remaining 20 \%. The correlation of predicted and actual probability of success suggests that classifier is well calibrated.}
    \label{fig:classifier_calibration}
\end{figure}

Furthermore, when the simulation success rate is plotted across different ControlBO configurations in Figure \ref{fig:mean_diff_classifier}, the classifier clearly improves the fraction of tuning experiments that are successful. Without the classifier, the majority of the simulations fail.  

% Comparison of success rate with and without classifier
\begin{figure}
    \centering
    \includegraphics[width=0.8\textwidth]{gfx/Chapter06/mean_diff_success_fraction.png}
    \caption{Fraction of suggestions that were successful during tuning campaigns when using various configurations of ControlBO. ST stands for single-task, MT for multi-task, C is for classifier and NC is for no classifier.}
    \label{fig:mean_diff_classifier}
\end{figure}


\subsection{Performance after tuning}

Figure \ref{fig:dynamic_after_tuning} shows the performance of the distillation column after tuning. Compared to the original tuning after parameter estimation shown in Figure \ref{fig:estimated}, the condenser and and sump levels are better controlled with tightly tuned loops. The temperature controller is slow to respond, but this is due to limitations of the simulation; when more aggressive controller parameters are used the simulation fails.

\begin{figure}
    \centering
    \includegraphics[width=\textwidth]{gfx/Chapter06/mt_c_dynamic.png}
    \caption{Dynamic simulation after controller tuning of a separation of 50/50 methanol-water separation in a distillation column with 80 bubble cap trays. \textbf{Top left}: Liquid composition over time. \textbf{Top right}: Temperature over time. \textbf{Bottom left}: Flow over time \textbf{Bottom right}: Holdups in condenser and reboiler.}
    \label{fig:dynamic_after_tuning}
\end{figure}

\section{Conclusions}

Here, I compared several approaches to optimising control systems of distillation columns using Bayesian optimisation. I developed a simulation of a distillation column, which I aligned to experiments conducted by BASF. Subsequently, I used this simulation to explore the use of ControlBO, a multi-fidelity, multi-objective Bayesian optimisation strategy designed for controller tuning. I found that ControlBO could find optimal tuning parameters within twenty experiments. Using ControlBO in multi-fidelity mode (i.e., with a multi-task model) improved the speed of optimisation but did not result in better final tuning. However, when a classifier was used to filter out suggestions that might fail, ControlBO gave better solutions on average.

This work is a first step in the direction of leveraging a combination of simulation and machine learning to accelerate controller tuning. However, there are significant challenges both in developing a robust dynamic simulation of a distillation column and applying Bayesian optimisation to controller tuning. My distillation simulation does not take into account dynamic pressure changes, which do occur in actual laboratory distillation columns. While the experimental data suggest that pressure drop changes are small in the specific case study due to the use of a vacuum pump for control, more general application of this approach would require taking into account these pressure dynamics to accurately represent dynamic behavior. Similarly, the Bayesian optimisation strategy only uses simple time-domain controller performance metrics, while frequency domain performance metrics might give more transferable results. Additionally, we test only on simulations here, but deployment on a real distillation column would be key to verifying the viability of this approach.

