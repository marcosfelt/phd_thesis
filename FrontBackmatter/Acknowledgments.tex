%*******************************************************
% Acknowledgments
%*******************************************************
\pdfbookmark[1]{Acknowledgments}{acknowledgments}

\chapter*{Acknowledgments}
When I arrived in Cambridge for a MPhil Research, I had no idea that I would be finishing  a PhD nearly five years later. I am grateful to Prof. Alexei Lapkin who saw in me something that I didn't see at the time; that a PhD was the right and best way to achieve my goals!

Prof. Lapkin has been a supportive and patient supervisor. I thank him for allowing me to explore my wide range of interests including startups. I am also grateful that he gracefully dealt with my simultaneously great ambitions and tendency to leave things to the last minute. He was the perfect supervisor for me.

I gratefully acknowledge the Cambridge Trust-Marshall Scholarship and BASF SE for funding my PhD. I would like to thank my collaborators at BASF, Julian Meyer-Kirschner and Carsten Kn\"osche, for their patience and commitment to achieving our goals. I would also like to express my gratitude to Christian Holtze for always focusing on how to translate the insights from my research to applications at BASF. I also would like to thank Prof. Alexander Mitsos and Prof. Kai Leonhard for facilitating my virtual stay at RWTH Aachen and providing essential feedback on the ML-SAFT work. 

I would like to thank the student and postdoc collaborators without whom this work would not have been possible. I am extremely grateful to Jan Rittig, who I first worked with on Summit when he was an intern our group in the summer of 2020 and subsequently in the last year of my PhD on ML-SAFT. His constant stream of great ideas, knowledge of the literature and ability to rapidly implement anything in code always amazes me. Daniel Wigh was an excellent collaborator and probably the only person who would ever say yes to my idea to code and write a conference paper on multitask Bayesian optimization in three days. I am forever grateful for his openness to new ideas and curiosity. I want to thank Connor Taylor for being an outstanding experimental collaborator and showing me how to be an efficient researcher. I thank Lukas Ra{\ss}pe-Lange whose insights into thermodynamics were invaluable for our paper on ML-SAFT and Hashem Ben-Safar who was an intern that helped me with \textit{DeepGamma}. Finally, I would like to thank Alexander Pomberger and Daniel Wigh for colloborating on a transfer learning project for reaction optimization, even if the outcome was different
than what we would have expected.

I would like to express my gratitude to the whole the Sustainable Reaction Engineering Group and SynTech CDT. I would like to give special thanks to Adarsh Arun for taking on all the initiatives that I started but could not continue, Ahmad Khan for our stimulating conversations, and Perman Jorayev for always making me laugh. I thank the SynTech CDT coordinators Mehrnaz and Chung for making our experience in the first years smooth.

I am grateful to the the people who helped me solve computational bottlenecks in my work. Jana Marie Weber showed me how to use the Cambridge cluster during my MPhil research, which was essential to scaling the experiments in the work on Summit. Robert Lee from Lightning.ai helped me debug the software that was used to scale the experiments for multitask Bayesian optimization and ML-SAFT work. 

I would like to thank the people who supported my endeavor with commercializing Bayesian reaction optimization. Thank you to  Chandler Gonzales  for embarking on the journey with me and being also open to shutting it down.  I am grateful to the staff at ConceptionX, the accelerator for PhD students, and Nick Russel and Patrick Gilday for providing mentorship.  

I am grateful to my friends for their support. I am thankful to my church family, especially Horton and Amanda who were like grandparents to me. Thank you to my housemates over the years in Cambridge, especially during the COVID-19 pandemic lockdowns. I would particularly like to thank my final year housemates Sofia and Chie who changed my perspective on how all parts of your life can be consistent. 

I would like to thank my family. My grandmothers, who both passed away during my PhD, were always my greatest cheerleaders. I most of all want to thank my parents who gave me immense opportunities and taught me to work hard and ask good questions.

Finally, thank you Ernie for always putting a smile on my face and helping me to get through the hardest times and celebrate the best ones.





